\documentclass{article}
\usepackage[includeheadfoot,top=20mm, bottom=20mm, footskip=2.5cm]{geometry}

% Typography
\usepackage[T1]{fontenc}
\usepackage{times}
%\usepackage{mathptmx} % math also in times font
\usepackage{amssymb,amsmath}
\usepackage{microtype}
\usepackage[utf8]{inputenc}

% Misc
\usepackage{graphicx}
\usepackage[hidelinks]{hyperref} %textopdfstring from pandoc

% Table

\usepackage{adjustbox} % center large tables across textwidth by surrounding tabular with \begin{adjustbox}{center}
\renewcommand{\arraystretch}{1.5} % enlarge spacing between rows
\usepackage{caption} 
\captionsetup[table]{skip=10pt} % enlarge spacing between caption and table

% Section styles

\usepackage{titlesec}
\titleformat{\section}{\normalfont\large}{\makebox[0pt][r]{\bf \thesection.\hspace{4mm}}}{0em}{\bfseries}
\titleformat{\subsection}{\normalfont}{\makebox[0pt][r]{\bf \thesubsection.\hspace{4mm}}}{0em}{\bfseries\vspace*{-3mm}}

% Paragraph styles

\setlength{\parskip}{0.6\baselineskip}%
\setlength{\parindent}{0pt}%

% Quotation styles

\usepackage{framed}
\let\oldquote=\quote
\let\endoldquote=\endquote
\renewenvironment{quote}{\begin{fquote}\advance\leftmargini -2.4em\begin{oldquote}}{\end{oldquote}\end{fquote}}

\usepackage{xcolor}
\newenvironment{fquote}
  {\def\FrameCommand{
	\fboxsep=0.6em % box to text padding
	\fcolorbox{black}{white}}%
	% the "2" can be changed to make the box smaller
    \MakeFramed {\advance\hsize-2\width \FrameRestore}
    \begin{minipage}{\linewidth}
  }
  {\end{minipage}\endMakeFramed}

% Table styles

\let\oldtabular=\tabular
\let\endoldtabular=\endtabular
\renewenvironment{tabular}[1]{\begin{adjustbox}{center}\begin{oldtabular}{#1}}{\end{oldtabular}\end{adjustbox}}


% Shortcuts

%% Let textbf be both, bold and italic
%\DeclareTextFontCommand{\textbf}{\bfseries\em}

%% Add RC and AR to the left of a paragraph
%\def\RC{\makebox[0pt][r]{\bf RC:\hspace{4mm}}}
%\def\AR{\makebox[0pt][r]{AR:\hspace{4mm}}}

%% Define that \RC and \AR should start and format the whole paragraph 
\usepackage{suffix}
\long\def\RC#1\par{\makebox[0pt][r]{\bf RC:\hspace{4mm}}\textbf{\textit{#1}}\par} %\RC
\WithSuffix\long\def\RC*#1\par{\textbf{\textit{#1}}\par} %\RC*
\long\def\AR#1\par{\makebox[0pt][r]{AR:\hspace{10pt}}\textit{#1}\par} %\AR
\WithSuffix\long\def\AR*#1\par{\textit{#1}\par} %\AR*


%%%
%DIF PREAMBLE EXTENSION ADDED BY LATEXDIFF
%DIF UNDERLINE PREAMBLE %DIF PREAMBLE
\RequirePackage[normalem]{ulem} %DIF PREAMBLE
\RequirePackage{color}\definecolor{RED}{rgb}{1,0,0}\definecolor{BLUE}{rgb}{0,0,1} %DIF PREAMBLE
\providecommand{\DIFadd}[1]{{\protect\color{blue}\uwave{#1}}} %DIF PREAMBLE
\providecommand{\DIFdel}[1]{{\protect\color{red}\sout{#1}}}                      %DIF PREAMBLE
%DIF SAFE PREAMBLE %DIF PREAMBLE
\providecommand{\DIFaddbegin}{} %DIF PREAMBLE
\providecommand{\DIFaddend}{} %DIF PREAMBLE
\providecommand{\DIFdelbegin}{} %DIF PREAMBLE
\providecommand{\DIFdelend}{} %DIF PREAMBLE
%DIF FLOATSAFE PREAMBLE %DIF PREAMBLE
\providecommand{\DIFaddFL}[1]{\DIFadd{#1}} %DIF PREAMBLE
\providecommand{\DIFdelFL}[1]{\DIFdel{#1}} %DIF PREAMBLE
\providecommand{\DIFaddbeginFL}{} %DIF PREAMBLE
\providecommand{\DIFaddendFL}{} %DIF PREAMBLE
\providecommand{\DIFdelbeginFL}{} %DIF PREAMBLE
\providecommand{\DIFdelendFL}{} %DIF PREAMBLE
%DIF END PREAMBLE EXTENSION ADDED BY LATEXDIFF


\begin{document}

\documentclass{article}
\usepackage[includeheadfoot,top=20mm, bottom=20mm, footskip=2.5cm]{geometry}

% Typography
\usepackage[T1]{fontenc}
\usepackage{times}
%\usepackage{mathptmx} % math also in times font
\usepackage{amssymb,amsmath}
\usepackage{microtype}
\usepackage[utf8]{inputenc}

% Misc
\usepackage{graphicx}
\usepackage[hidelinks]{hyperref} %textopdfstring from pandoc

% Table

\usepackage{adjustbox} % center large tables across textwidth by surrounding tabular with \begin{adjustbox}{center}
\renewcommand{\arraystretch}{1.5} % enlarge spacing between rows
\usepackage{caption} 
\captionsetup[table]{skip=10pt} % enlarge spacing between caption and table

% Section styles

\usepackage{titlesec}
\titleformat{\section}{\normalfont\large}{\makebox[0pt][r]{\bf \thesection.\hspace{4mm}}}{0em}{\bfseries}
\titleformat{\subsection}{\normalfont}{\makebox[0pt][r]{\bf \thesubsection.\hspace{4mm}}}{0em}{\bfseries\vspace*{-3mm}}

% Paragraph styles

\setlength{\parskip}{0.6\baselineskip}%
\setlength{\parindent}{0pt}%

% Quotation styles

\usepackage{framed}
\let\oldquote=\quote
\let\endoldquote=\endquote
\renewenvironment{quote}{\begin{fquote}\advance\leftmargini -2.4em\begin{oldquote}}{\end{oldquote}\end{fquote}}

\usepackage{xcolor}
\newenvironment{fquote}
  {\def\FrameCommand{
	\fboxsep=0.6em % box to text padding
	\fcolorbox{black}{white}}%
	% the "2" can be changed to make the box smaller
    \MakeFramed {\advance\hsize-2\width \FrameRestore}
    \begin{minipage}{\linewidth}
  }
  {\end{minipage}\endMakeFramed}

% Table styles

\let\oldtabular=\tabular
\let\endoldtabular=\endtabular
\renewenvironment{tabular}[1]{\begin{adjustbox}{center}\begin{oldtabular}{#1}}{\end{oldtabular}\end{adjustbox}}


% Shortcuts

%% Let textbf be both, bold and italic
%\DeclareTextFontCommand{\textbf}{\bfseries\em}

%% Add RC and AR to the left of a paragraph
%\def\RC{\makebox[0pt][r]{\bf RC:\hspace{4mm}}}
%\def\AR{\makebox[0pt][r]{AR:\hspace{4mm}}}

%% Define that \RC and \AR should start and format the whole paragraph 
\usepackage{suffix}
\long\def\RC#1\par{\makebox[0pt][r]{\bf RC:\hspace{4mm}}\textbf{\textit{#1}}\par} %\RC
\WithSuffix\long\def\RC*#1\par{\textbf{\textit{#1}}\par} %\RC*
\long\def\AR#1\par{\makebox[0pt][r]{AR:\hspace{10pt}}\textit{#1}\par} %\AR
\WithSuffix\long\def\AR*#1\par{\textit{#1}\par} %\AR*


%%%
%DIF PREAMBLE EXTENSION ADDED BY LATEXDIFF
%DIF UNDERLINE PREAMBLE %DIF PREAMBLE
\RequirePackage[normalem]{ulem} %DIF PREAMBLE
\RequirePackage{color}\definecolor{RED}{rgb}{1,0,0}\definecolor{BLUE}{rgb}{0,0,1} %DIF PREAMBLE
\providecommand{\DIFadd}[1]{{\protect\color{blue}\uwave{#1}}} %DIF PREAMBLE
\providecommand{\DIFdel}[1]{{\protect\color{red}\sout{#1}}}                      %DIF PREAMBLE
%DIF SAFE PREAMBLE %DIF PREAMBLE
\providecommand{\DIFaddbegin}{} %DIF PREAMBLE
\providecommand{\DIFaddend}{} %DIF PREAMBLE
\providecommand{\DIFdelbegin}{} %DIF PREAMBLE
\providecommand{\DIFdelend}{} %DIF PREAMBLE
%DIF FLOATSAFE PREAMBLE %DIF PREAMBLE
\providecommand{\DIFaddFL}[1]{\DIFadd{#1}} %DIF PREAMBLE
\providecommand{\DIFdelFL}[1]{\DIFdel{#1}} %DIF PREAMBLE
\providecommand{\DIFaddbeginFL}{} %DIF PREAMBLE
\providecommand{\DIFaddendFL}{} %DIF PREAMBLE
\providecommand{\DIFdelbeginFL}{} %DIF PREAMBLE
\providecommand{\DIFdelendFL}{} %DIF PREAMBLE
%DIF END PREAMBLE EXTENSION ADDED BY LATEXDIFF


\begin{document}

\documentclass{article}
\usepackage[includeheadfoot,top=20mm, bottom=20mm, footskip=2.5cm]{geometry}

% Typography
\usepackage[T1]{fontenc}
\usepackage{times}
%\usepackage{mathptmx} % math also in times font
\usepackage{amssymb,amsmath}
\usepackage{microtype}
\usepackage[utf8]{inputenc}

% Misc
\usepackage{graphicx}
\usepackage[hidelinks]{hyperref} %textopdfstring from pandoc

% Table

\usepackage{adjustbox} % center large tables across textwidth by surrounding tabular with \begin{adjustbox}{center}
\renewcommand{\arraystretch}{1.5} % enlarge spacing between rows
\usepackage{caption} 
\captionsetup[table]{skip=10pt} % enlarge spacing between caption and table

% Section styles

\usepackage{titlesec}
\titleformat{\section}{\normalfont\large}{\makebox[0pt][r]{\bf \thesection.\hspace{4mm}}}{0em}{\bfseries}
\titleformat{\subsection}{\normalfont}{\makebox[0pt][r]{\bf \thesubsection.\hspace{4mm}}}{0em}{\bfseries\vspace*{-3mm}}

% Paragraph styles

\setlength{\parskip}{0.6\baselineskip}%
\setlength{\parindent}{0pt}%

% Quotation styles

\usepackage{framed}
\let\oldquote=\quote
\let\endoldquote=\endquote
\renewenvironment{quote}{\begin{fquote}\advance\leftmargini -2.4em\begin{oldquote}}{\end{oldquote}\end{fquote}}

\usepackage{xcolor}
\newenvironment{fquote}
  {\def\FrameCommand{
	\fboxsep=0.6em % box to text padding
	\fcolorbox{black}{white}}%
	% the "2" can be changed to make the box smaller
    \MakeFramed {\advance\hsize-2\width \FrameRestore}
    \begin{minipage}{\linewidth}
  }
  {\end{minipage}\endMakeFramed}

% Table styles

\let\oldtabular=\tabular
\let\endoldtabular=\endtabular
\renewenvironment{tabular}[1]{\begin{adjustbox}{center}\begin{oldtabular}{#1}}{\end{oldtabular}\end{adjustbox}}


% Shortcuts

%% Let textbf be both, bold and italic
%\DeclareTextFontCommand{\textbf}{\bfseries\em}

%% Add RC and AR to the left of a paragraph
%\def\RC{\makebox[0pt][r]{\bf RC:\hspace{4mm}}}
%\def\AR{\makebox[0pt][r]{AR:\hspace{4mm}}}

%% Define that \RC and \AR should start and format the whole paragraph 
\usepackage{suffix}
\long\def\RC#1\par{\makebox[0pt][r]{\bf RC:\hspace{4mm}}\textbf{\textit{#1}}\par} %\RC
\WithSuffix\long\def\RC*#1\par{\textbf{\textit{#1}}\par} %\RC*
\long\def\AR#1\par{\makebox[0pt][r]{AR:\hspace{10pt}}\textit{#1}\par} %\AR
\WithSuffix\long\def\AR*#1\par{\textit{#1}\par} %\AR*


%%%
%DIF PREAMBLE EXTENSION ADDED BY LATEXDIFF
%DIF UNDERLINE PREAMBLE %DIF PREAMBLE
\RequirePackage[normalem]{ulem} %DIF PREAMBLE
\RequirePackage{color}\definecolor{RED}{rgb}{1,0,0}\definecolor{BLUE}{rgb}{0,0,1} %DIF PREAMBLE
\providecommand{\DIFadd}[1]{{\protect\color{blue}\uwave{#1}}} %DIF PREAMBLE
\providecommand{\DIFdel}[1]{{\protect\color{red}\sout{#1}}}                      %DIF PREAMBLE
%DIF SAFE PREAMBLE %DIF PREAMBLE
\providecommand{\DIFaddbegin}{} %DIF PREAMBLE
\providecommand{\DIFaddend}{} %DIF PREAMBLE
\providecommand{\DIFdelbegin}{} %DIF PREAMBLE
\providecommand{\DIFdelend}{} %DIF PREAMBLE
%DIF FLOATSAFE PREAMBLE %DIF PREAMBLE
\providecommand{\DIFaddFL}[1]{\DIFadd{#1}} %DIF PREAMBLE
\providecommand{\DIFdelFL}[1]{\DIFdel{#1}} %DIF PREAMBLE
\providecommand{\DIFaddbeginFL}{} %DIF PREAMBLE
\providecommand{\DIFaddendFL}{} %DIF PREAMBLE
\providecommand{\DIFdelbeginFL}{} %DIF PREAMBLE
\providecommand{\DIFdelendFL}{} %DIF PREAMBLE
%DIF END PREAMBLE EXTENSION ADDED BY LATEXDIFF


\begin{document}

\documentclass{article}
\usepackage[includeheadfoot,top=20mm, bottom=20mm, footskip=2.5cm]{geometry}

% Typography
\usepackage[T1]{fontenc}
\usepackage{times}
%\usepackage{mathptmx} % math also in times font
\usepackage{amssymb,amsmath}
\usepackage{microtype}
\usepackage[utf8]{inputenc}

% Misc
\usepackage{graphicx}
\usepackage[hidelinks]{hyperref} %textopdfstring from pandoc

% Table

\usepackage{adjustbox} % center large tables across textwidth by surrounding tabular with \begin{adjustbox}{center}
\renewcommand{\arraystretch}{1.5} % enlarge spacing between rows
\usepackage{caption} 
\captionsetup[table]{skip=10pt} % enlarge spacing between caption and table

% Section styles

\usepackage{titlesec}
\titleformat{\section}{\normalfont\large}{\makebox[0pt][r]{\bf \thesection.\hspace{4mm}}}{0em}{\bfseries}
\titleformat{\subsection}{\normalfont}{\makebox[0pt][r]{\bf \thesubsection.\hspace{4mm}}}{0em}{\bfseries\vspace*{-3mm}}

% Paragraph styles

\setlength{\parskip}{0.6\baselineskip}%
\setlength{\parindent}{0pt}%

% Quotation styles

\usepackage{framed}
\let\oldquote=\quote
\let\endoldquote=\endquote
\renewenvironment{quote}{\begin{fquote}\advance\leftmargini -2.4em\begin{oldquote}}{\end{oldquote}\end{fquote}}

\usepackage{xcolor}
\newenvironment{fquote}
  {\def\FrameCommand{
	\fboxsep=0.6em % box to text padding
	\fcolorbox{black}{white}}%
	% the "2" can be changed to make the box smaller
    \MakeFramed {\advance\hsize-2\width \FrameRestore}
    \begin{minipage}{\linewidth}
  }
  {\end{minipage}\endMakeFramed}

% Table styles

\let\oldtabular=\tabular
\let\endoldtabular=\endtabular
\renewenvironment{tabular}[1]{\begin{adjustbox}{center}\begin{oldtabular}{#1}}{\end{oldtabular}\end{adjustbox}}


% Shortcuts

%% Let textbf be both, bold and italic
%\DeclareTextFontCommand{\textbf}{\bfseries\em}

%% Add RC and AR to the left of a paragraph
%\def\RC{\makebox[0pt][r]{\bf RC:\hspace{4mm}}}
%\def\AR{\makebox[0pt][r]{AR:\hspace{4mm}}}

%% Define that \RC and \AR should start and format the whole paragraph 
\usepackage{suffix}
\long\def\RC#1\par{\makebox[0pt][r]{\bf RC:\hspace{4mm}}\textbf{\textit{#1}}\par} %\RC
\WithSuffix\long\def\RC*#1\par{\textbf{\textit{#1}}\par} %\RC*
\long\def\AR#1\par{\makebox[0pt][r]{AR:\hspace{10pt}}\textit{#1}\par} %\AR
\WithSuffix\long\def\AR*#1\par{\textit{#1}\par} %\AR*


%%%
%DIF PREAMBLE EXTENSION ADDED BY LATEXDIFF
%DIF UNDERLINE PREAMBLE %DIF PREAMBLE
\RequirePackage[normalem]{ulem} %DIF PREAMBLE
\RequirePackage{color}\definecolor{RED}{rgb}{1,0,0}\definecolor{BLUE}{rgb}{0,0,1} %DIF PREAMBLE
\providecommand{\DIFadd}[1]{{\protect\color{blue}\uwave{#1}}} %DIF PREAMBLE
\providecommand{\DIFdel}[1]{{\protect\color{red}\sout{#1}}}                      %DIF PREAMBLE
%DIF SAFE PREAMBLE %DIF PREAMBLE
\providecommand{\DIFaddbegin}{} %DIF PREAMBLE
\providecommand{\DIFaddend}{} %DIF PREAMBLE
\providecommand{\DIFdelbegin}{} %DIF PREAMBLE
\providecommand{\DIFdelend}{} %DIF PREAMBLE
%DIF FLOATSAFE PREAMBLE %DIF PREAMBLE
\providecommand{\DIFaddFL}[1]{\DIFadd{#1}} %DIF PREAMBLE
\providecommand{\DIFdelFL}[1]{\DIFdel{#1}} %DIF PREAMBLE
\providecommand{\DIFaddbeginFL}{} %DIF PREAMBLE
\providecommand{\DIFaddendFL}{} %DIF PREAMBLE
\providecommand{\DIFdelbeginFL}{} %DIF PREAMBLE
\providecommand{\DIFdelendFL}{} %DIF PREAMBLE
%DIF END PREAMBLE EXTENSION ADDED BY LATEXDIFF


\begin{document}

\input{RRL.tex}

\end{document}

\end{document}

\end{document}

\end{document}